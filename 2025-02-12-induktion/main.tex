\documentclass{article}

\usepackage[a4paper]{geometry}
\usepackage[utf8]{inputenc}
\usepackage[T1]{fontenc}
\usepackage{lmodern}
\usepackage[UKenglish]{babel} %[british, UKenglish, USenglish, english, american]

\usepackage{amsmath}
\usepackage{amssymb}
\usepackage{amsthm}

\usepackage{fancyhdr}
\pagestyle{fancy}
%\geometry{verbose,tmargin=3cm,bmargin=3cm, lmargin=2.5cm, rmargin=2.5cm,headheight=1.5cm,headsep=1.5cm}

\usepackage{graphicx}
\usepackage{float}
\usepackage[hidelinks]{hyperref}

\usepackage{lastpage}
\usepackage{listings}

\usepackage{tikz}

\usepackage{enumitem}

\usepackage{dsfont}
\usepackage{wrapfig}
\usepackage{caption}
\usepackage{color}
\usepackage{textcomp}
\usepackage{mdframed}

\usepackage{polynom}
\usepackage{forest}
\usepackage{qtree}
\usepackage{graphics,graphicx}
\usepackage{pstricks,pst-node,pst-tree}

\numberwithin{equation}{section}

\theoremstyle{plain}
\newtheorem{saetning}{Sætning}[section]
\newtheorem{lemma}[saetning]{Lemma}
\newtheorem{korollar}[saetning]{Korollar}
\newtheorem{proposition}[saetning]{Proposition}

\theoremstyle{definition}
\newtheorem{definition}[saetning]{Definition}
\newtheorem{eksempel}[saetning]{Eksempel}
\newtheorem{notation}[saetning]{Notation}


%\setlength\parindent{0pt} % ingen indention ved linjeskift
\linespread{1.5} % linjeafstand

\lhead{Department of Computer Science and Mathematics}
\chead{}
\rhead{University of Southern Denmark}
\cfoot{\thepage\ of \pageref{LastPage}}


\begin{document}

\title{UNF Odense Matematikklub: Induktion}
\author{Hjalte Düsterdich Vejbæk\\ \href{mailto:hdv@unf.dk}{hdv@unf.dk}\and Rasmus Hauge Hansen\\ \href{mailto:rhh@unf.dk}{rhh@unf.dk}}
\date{\today}
\maketitle{}
\thispagestyle{empty}

%start på dokument
\newpage

\section{Hvad er induktion?}
    Vi spørger Ordnet:
    \begin{quote}
        \textit{``fremkaldelse af elektrisk spænding i fx en ledning eller en spole ved ændring af det magnetfelt der omgiver denne''}
    \end{quote}
    det er måsek ikke helt rigtig, vi prøver den næste definition
    \begin{quote}
        \textit{``fremkaldelse eller stimulering af en ændring, reaktion el.lign. især vedr. biologiske eller biokemiske forhold''}
    \end{quote}
    heller ikke helt rigtigt. Hvad med den sidste
    \begin{quote}
        \textit{``logisk tænkemåde hvor man ud fra enkelttilfælde slutter sig til en generel regel, en lovmæssighed eller en hypotese''}
    \end{quote}
    Det virker mere rigtigt. Dog lyder det her ikke helt for mig som matematiker. Jeg kan nemlig godt lide at være sten sikker på at min ``lovmæssighed'' eller ``hypotese'' er helt rigtig. Altså jeg tror generelt ikke på, at tilstrækkelig mange observationer er godt nok til at lave en generel regel. Lad mig komme med et eksempel
    \begin{eksempel}[Russel]
        Vi møder en kalkun d. 1. januar. Den lever på en amerikansk gård. Den første dag kommer bondemanden ind og giver kalkunen mad. Vi kan på nuværende tidspunkt ikke konkluderer noget. Det samme sker d. 2. januar; vi kan stadig ikke sige noget. Dette fortsætter helt ind til juli måned. Vi laver nu den logisk induktive slutning: ``hver dag kommer bondemanden og giver kalkunen mad''. Dette fortsætter ganske vist gennem juli, august, september, oktober og helt end til d. 4. torsdag i november, hvor bondemanden kommer og knækker narken på kalkunen.
    \end{eksempel}
    Konklusion logisk induktion er ikke godt nok til matematik, dog ville det være rimelig nice, hvis vi kunne benytte ideen af at flere observationer kan føre til en generel regel.
    
\section{Matematisk induktion}
    Vi vil derfor gerne udvikle vores ``egen induktion''. Lad os kalde det ``matematisk induktion''.
\end{document}
