\documentclass[a4paper, 12pt]{article}

\usepackage{geometry}
\usepackage[utf8]{inputenc}
\usepackage[T1]{fontenc}
\usepackage{lmodern}
\usepackage[danish]{babel} %[british, UKenglish, USenglish, english, american]

\usepackage{amsmath}
\usepackage{amssymb}
\usepackage{amsthm}

\usepackage{fancyhdr}
\pagestyle{fancy}
%\geometry{verbose,tmargin=3cm,bmargin=3cm, lmargin=2.5cm, rmargin=2.5cm,headheight=1.5cm,headsep=1.5cm}

\usepackage{graphicx}
\usepackage{float}
\usepackage[hidelinks]{hyperref}
\usepackage{cleveref}

\usepackage{lastpage}
\usepackage{listings}


\usepackage{enumitem}
\numberwithin{equation}{section}
\theoremstyle{plain}
\newtheorem{saetning}{Sætning}[section]
\newtheorem{lemma}[saetning]{Lemma}
\newtheorem{korollar}[saetning]{Korollar}
\newtheorem{proposition}[saetning]{Proposition}

\theoremstyle{definition}
\newtheorem{definition}[saetning]{Definition}
\newtheorem{eksempel}[saetning]{Eksempel}
\newtheorem{notation}[saetning]{Notation}

\crefname{equation}{ligning}{ligninger}
\newcommand{\N}{\mathbb{N}}


%\setlength\parindent{0pt} % ingen indention ved linjeskift

\lhead{UNF Odense Matematikklub}
\chead{}
\rhead{Matematisk Induktion}
\cfoot{\thepage\ af \pageref{LastPage}}


\title{UNF Odense Matematikklub: Induktion}
\author{Hjalte Düsterdich Vejbæk\\ \href{mailto:hdv@unf.dk}{hdv@unf.dk}\and Rasmus Hauge Hansen\\ \href{mailto:rhh@unf.dk}{rhh@unf.dk}}
\date{\today}
\begin{document}

\maketitle{}
\thispagestyle{empty}

%start på dokument
\newpage

\section{Hvad er induktion?}
Hvis vi skal arbejde med induktion, må vi først vide hvad det er. Vi spørger Ordnet:
    \begin{quote}
        \textit{``Fremkaldelse af elektrisk spænding i fx en ledning eller en spole ved ændring af det magnetfelt der omgiver denne.''}
    \end{quote}
    Det er måske ikke helt rigtig, vi prøver den næste definition.
    \begin{quote}
        \textit{``Fremkaldelse eller stimulering af en ændring, reaktion el.lign. især vedr. biologiske eller biokemiske forhold.''}
    \end{quote}
    Heller ikke helt det vi leder efter. Hvad med den sidste?
    \begin{quote}
        \textit{``Logisk tænkemåde hvor man ud fra enkelttilfælde slutter sig til en generel regel, en lovmæssighed eller en hypotese.''}
    \end{quote}
    Det virker mere rigtigt. Dog fungerer det her ikke helt for os som matematikere. Vi kan nemlig godt lide at være stensikre på at vores ``lovmæssighed'' eller ``hypotese'' er helt rigtig. Altså tror vi generelt ikke på, at tilstrækkelig mange observationer er godt nok til at lave en generel regel. Lad os kigge på et eksempel.
    \begin{eksempel}[Russel]
        Vi møder en kalkun d. 1. januar. Den lever på en amerikansk gård. Den første dag kommer bondemanden ind og giver kalkunen mad. Vi kan på nuværende tidspunkt ikke konkludere noget. Det samme sker d. 2. januar; vi kan stadig ikke sige noget. Dette fortsætter helt indtil juli måned. Vi laver nu den logisk induktive slutning: ``hver dag kommer bondemanden og giver kalkunen mad''. Dette fortsætter ganske vist gennem juli, august, september, oktober og helt indtil d. 4. torsdag i november, hvor bondemanden kommer og knækker nakken på kalkunen.
    \end{eksempel}
    Vi må konkludere, at logisk induktion ikke er godt nok til matematik. Dog ville det være rimelig nice, hvis vi kunne benytte ideen af at flere observationer kan føre til en generel regel til at bevise noget matematik.
\section{Lidt om de naturlige tal}
Vi vil nu udvikle vores ``egen'' induktion, som vi ønsker at kunne bruge til at bevise matematiske sætninger. Vi vil kalde vores spritnye induktion for ``matematisk induktion'', og hvis vi ønsker at kunne lave beviser med den, må den altså være en del mere stærk end hvad vi oplevede i sidste sektion.

Vi starter med at indføre en særlig mængde, som spiller en vigtig rolle for induktionen.
\begin{definition}[Naturlige tal]
  De \emph{naturlige tal}, skrevet \(\N\), er mængden af alle positive heltal. Altså,
  \[
    \N:=\{1,2,3,\ldots\}.
  \]
  Desuden definerer vi \(\N_0\) som de naturlige tal \emph{inklusiv 0} eller de \emph{ikke-negative heltal}, altså
  \[
    \N_0:=\{0,1,2,\ldots\}.
    \]
\end{definition}
Som matematikere kan vi godt lide at have meget konkrete og præcise definitioner, og hvis du viste ovenstående definition til den almægtige Gau{\ss}, ville han sandsynligvis fordømme de næste 10 generationer i din familie til et liv uden matematik. Der findes forskellige måder at definere de ikke-negative heltal på, bl.a. med mængdelære, men det er ikke dagens mål. Vores ``definition'' er nok til at forstå matematisk induktion, hvilket til gengæld er dagens mål.

Vores matematiske induktion kommer til at gøre brug af to grundlæggende fakta om de ikke-negative heltal, som vi her vil tage for givet. Det første princip er at hvis vi har et  tal \(n\in\N_0\), så er der et entydigt ``næste tal,'' netop \(n+1\). På samme måde har \(n\) også et entydigt ``forrige tal,'' altså hvis \(n\neq 0\).

Hvis vi skriver \(S(n)\) for ``tallet efter \(n\)'', så har vi særligt at ethvert tal \(m\in\N_0\) er på formen
\[
  \underbrace{S(S(\ldots S}_{m\text{ gange}}(0)\ldots)),
\]
hvilket vil sige at vi kan nå til ethvert positivt heltal ved at starte ved 0 og så gå videre til det næste tal nok gange. Det gør 0 til et slags ``basistal'' for \(\N_0\).

Det andet princip vi vil bruge er \emph{velordningsprincippet}
\begin{proposition}[Velordningsprincippet]
  Enhver delmængde \(M\subseteq \N_0\) har et \emph{mindste element}, altså et element \(m\in M\) således at \(m\le k\) for alle \(k\in M\).
\end{proposition}
Hvis \(m\) er det mindste element i \(M\subseteq \N_0\), og vi desuden har \(k\in M\), så er
\[
  k=\underbrace{S(S(\ldots S}_{k\text{ gange}}(0)\ldots ))=\underbrace{S(S(\ldots S}_{k-m\text{ gange}}(m)\ldots )),
  \]
  og altså kan vi ``ramme'' alle \(k\in M\) ved at starte fra \(m\) og gå videre til det næste tal i rækkefølgen et tilstrækkeligt antal gange, så \(m\) er altså det nye ``basistal'' i vores nye mængde \(M\), og erstatter 0's rolle. Hvis man grubler længe nok over disse observationer kan man måske komme frem til en nyttig måde at anvende dem på til at bevise matematiske sætninger om \(\N\) og \(\N_0\); vi er endelig klar til at sætte de matematiske magneter i gang.
\section{Matematisk induktion}
Nu er den her endelig! Bevismetoden vi alle har ventet på! \emph{INDUKTION!!!}. Vi introducerer metoden med et eksempel. Vi vil gerne vise at 
\[
  2m<m^2
\]
gælder for alle ikke-negative heltal \(m\ge 3\). Vi kan jo starte med at tjekke at udsagnet holder for \(m=3\). 3 er her er vores ``basistal'' for mængden \(M\) af naturlige tal større end eller lig 3. Hvis ikke vi kan vise uligheden for \(m=3\) har vi desuden i hvert fald intet håb om at udsagnet kan være sandt for alle \(m\ge 3\). For \(m=3\) har vi
\[
  2m=6,\ m^2=9.
  \]
  Altså er udsagnet sandt for \(m=3\), da \(6<9\). Ved at verficere at det holder for \(m=3\) har vi allerede orndet første trin i induktionsbeviset. Dette trin kaldes \emph{induktionsbasis.}

  Vi kunne fortsætte med at bevise udsagnet for \(m=4,5,6,\ldots\), men så ville vi aldrig blive færdige -- og vi vil jo gerne også have tid til at lave andre induktionsbeviser, så det går ikke. Vi må altså finde på en anden fremgang.

  Lad os i stedet nu antage at udsagnet er sandt for et eller andet \(m\ge 3\) -- vi laver ikke nogen antagelser om hvad \(m\) er, andet end at det er større end eller lig 3. Dette kaldes \emph{induktionshypotesen}.

  Vi er nu næsten færdige -- vi mangler kun \emph{induktionsskridtet}. Når vi nu har antaget at udsagnet holder for \(m\), så vil vi vise at det holder for \(m+1\). Så lad os kigge på det
  \[
    2(m+1)=2m+2\buildrel{2m<m^2}\over{<}m^2+2=(m+1)^2+1-2m\buildrel{2m\ge 6>1}\over <(m+1)^2.
    \]
    Bemærk at vi her i den første ulighed har brugt induktionshypotesen, \(2m<m^2\) samt at vi har brugt at \(m\ge 3\). Vi har nu vist at
    \begin{enumerate}
    \item \(2m<m^2\) når \(m=3\);
      \item Hvis \(2m<m^2\) for et \(m\ge 3\), så er \(2(m+1)<(m+1)^2\).
    \end{enumerate}
    Vi påstår at vi nu har vist at \(2m<m^2\) for alle naturlige tal \(m\ge 3\), fordi hvis \(m\ge 3\), så er
    \[
      m=\underbrace{S(S(\ldots S}_{m-3\text{ gange}}(3)\ldots))=3+\underbrace{1+1+\ldots+1}_{m-3}.
      \]
      Men da udsagnet gælder for 3, må det gælde for \(3+1\) pr. induktionsskridtet, og dermed også \((3+1)+1\), et cetera, og vi kan fortsætte den kæde hele vejen op til \(m\), uanset hvor stor \(m\) er. Vi benytter os af, at der er et ``basistal,'' som udsagnet skal gælde for, og så udnytter vi at vi bare kan udtrykke alle andre tal som masser af iterationer af \(S\) anvendt på dette basistal.

      Det kan sammenlignes med at klatre op ad en stige: Hvis vi tager det første skridt op, og hvis det at vi tager et skridt op af stigen medfører at vi tager et skridt til, så når vi et hvilket som helst trin på stigen på et eller andet tidspunkt.

      Vi kan tage proceduren ovenfor og generalisere den til en metode til at bevise udsagn om vilkårlige delmængder af naturlige tal.
\begin{definition}[Matematisk induktion]
  Lad \(m\in\N_0\). Lad desuden \(P(n)\) være et logisk udsagn, som afhænger af et enkelt ikke-negativt heltal. At bruge \emph{matematisk induktion} til at vise at \(P(n)\) gælder for alle \(n\in\N_0\) med \(n\ge m\) består af følgende procedure:
  \begin{description}
  \item[Induktionsbasis:] Vis at \(P(m)\) holder.
  \item[Induktionshypotese:] Antag at \(P(n)\) holder for et eller andet \(n\in \N_0\) med \(n\ge m\).
    \item[Induktionsskridt:] Brug induktionshypotesen (samt evt. at vi ved at \(n\ge m\) i hypotesen) til at vise at \(P(n+1)\) er sandt.
  \end{description}
\end{definition}
Når man har udført de tre skridt i induktion, så har man vist \(P(n)\) for alle \(n\ge m\). Vi kan godt se os frem til hvorfor det virker: Lad \(M\subseteq \N_0\) være mængden af alle ikke-negative heltal \(n\), som er større end eller lig med \(m\). Det er tydeligt at \(m\) er det mindste element i \(M\), og vi har vist at \(P(m)\) er sandt. Som vi argumenterede for i vores tidligere eksempel har vi for alle \(n\in M\) at
\[
  n=m+\underbrace{1+1+\ldots+1}_{n-m\text{ gange}}.
\]
Men fordi \(P(m)\) er sandt er \(P(m+1)\) også sandt på grund af induktionsskridtet, og så falder dominobrikkerne, fordi vi så kan bruge induktionsskridtet på \(m+1\) for at få at \(P(m+2)\) er sandt, og så videre. Bevismetoden beror altså på at vi kun skal ``gå tilbage'' endeligt mange gange fra et vilkårligt tal \(n\ge m\) for at nå basistallet i \(M\).

Vi kan selvfølgelig også bruge induktion til at vise at noget holder for en delmængde \(M\) af \(\N_0\) som ikke bare består af alle tal større end et eller andet tal \(m\in\N_0\). Det er bare lidt overkill, for hvis der er ``huller'' mellem tallene i \(M\), så ender vi med at bevise vores udsagn for flere tal end bare dem i \(M\). Det samme sker hvis \(M\) ikke er en endelig mængde. Magten i induktion er at vi kan vise noget om den uendelige mængde \(\N_0\) ved at tage uendeligt mange skridt, men uden at sidde og bruge uendeligt lang tid på at lave beviset.
\section{Et eksempel}
Vi slutter af med endnu et eksempel på induktion, hvor vi bruger teknikken til at bevise en relativt velkendt og rimelig nyttig sætning.
\begin{saetning}
  Lad \(n\in\N\). Det gælder at
  \begin{equation}\label{eq:sumn}
    1+2+3+\dots+n=\frac{n(n+1)}{2}.
  \end{equation}
  Med venstresiden af ligheden menes summen af de første \(n\) heltal startende fra 1.
\end{saetning}
\begin{proof}[Bevis]
Bemærk at \(\N\subseteq\N_0\) bare er mængden af ikke-negative heltal som er større end 0, så det er tilstrækkeligt at vise at \cref{eq:sumn} gælder for alle \(n\in\N_0\) med \(n\ge 1\) ved induktion, så det gør vi.
\begin{description}
\item[Induktionsbasis:] Hvis vi betragter \(n=1\), så får vi
  \[
    \frac{n(n+1)}{2}=\frac{1\cdot 2}{2}=1,
    \]
    og hvis vi tager summen af de første 1 heltal startende fra 1, så får vi også 1. Altså holder \cref{eq:sumn} for \(n=1\), og vi har vores basis.
  \item[Induktionshypotese:] Vi antager at \(n\in\N_0\) er et tal så \(n\ge 1\) og \cref{eq:sumn} gælder for n.
  \item[Induktionsskridt:] Vi viser nu, på baggrund af vores antagelse, at \cref{eq:sumn} gælder for \(n+1\). Vi har at
    \[
      1+2+\dots+(n+1)=1+2+\dots+n+(n+1)=\frac{n(n+1)}{2}+(n+1).
      \]
    Vi kan nu sætte de to udtryk i den nye sum på fælles nævner, så vi får
    \begin{align*}
      1+2+\dots+(n+1)&=\frac{n(n+1)+2(n+1)}{2}\\
      &=\frac{(n+2)(n+1)}{2}=\frac{(n+1)(n+2)}{2}.
    \end{align*}
    Det er jo præcis det udtryk vi gerne ville have, så vi har nu gennemført induktionsskridtet.
\end{description}
Vi har nu gennemført alle skridtene i induktionsbeviset, og har altså vist at \cref{eq:sumn} gælder for alle \(n\in\N_0\) med \(n\ge 1\), hvilket er det samme som at vise at det gælder for alle \(n\in\N\), og vi er færdige :)
\end{proof}
Nu kan du absolut destruere din matematiklærer ved at regne \(1+2+\dots+100\) på 2 sekunder. Nemt.
\end{document}
