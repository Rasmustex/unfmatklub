\documentclass[a4paper, 12pt]{article}

\usepackage{geometry}
\usepackage[utf8]{inputenc}
\usepackage[T1]{fontenc}
\usepackage{lmodern}
\usepackage[danish]{babel} %[british, UKenglish, USenglish, english, american]

\usepackage{amsmath}
\usepackage{amssymb}
\usepackage{amsthm}
\usepackage{mathtools}

\usepackage{fancyhdr}
\pagestyle{fancy}
%\geometry{verbose,tmargin=3cm,bmargin=3cm, lmargin=2.5cm, rmargin=2.5cm,headheight=1.5cm,headsep=1.5cm}

\usepackage{graphicx}
\usepackage{float}
\usepackage[hidelinks]{hyperref}
\usepackage{cleveref}

\usepackage{lastpage}
\usepackage{listings}


\usepackage{enumitem}
\numberwithin{equation}{section}
\theoremstyle{plain}
\newtheorem{saetning}{Sætning}[section]
\newtheorem{lemma}[saetning]{Lemma}
\newtheorem{korollar}[saetning]{Korollar}
\newtheorem{proposition}[saetning]{Proposition}

\theoremstyle{definition}
\newtheorem{definition}[saetning]{Definition}
\newtheorem{eksempel}[saetning]{Eksempel}
\newtheorem{notation}[saetning]{Notation}

\crefname{equation}{ligning}{ligninger}
\DeclarePairedDelimiter{\ps}{(}{)}
\newcommand{\N}{\mathbb{N}}
\newcommand{\R}{\mathbb{R}}

\makeatletter
\renewcommand*\env@matrix[1][*\c@MaxMatrixCols c]{%
  \hskip -\arraycolsep
  \let\@ifnextchar\new@ifnextchar
  \array{#1}}
\makeatother

\usepackage{nicematrix}

%\setlength\parindent{0pt} % ingen indention ved linjeskift

\lhead{UNF Odense Matematikklub}
\chead{}
\rhead{Matematisk Induktion}
\cfoot{\thepage\ af \pageref{LastPage}}


\title{UNF Odense Matematikklub: Lineær algebra}
\author{Hjalte Vejbæk\\ \href{mailto:hdv@unf.dk}{hdv@unf.dk}\and Rasmus Hansen\\ \href{mailto:rhh@unf.dk}{rhh@unf.dk}\and Victor Heeks\\ \href{mailto:syhe@unf.dk}{syhe@unf.dk}}
\date{}


\begin{document}
\NiceMatrixOptions
  {
    code-for-first-col = \scriptstyle ,
    code-for-first-row = \scriptstyle 
  }

\maketitle{}
\thispagestyle{empty}

%start på dokument
\newpage
%%%%%%%%% TODO: Ordn pagenum counter på en andne måde
\setcounter{page}{1}
\section{Systemer af lineære ligninger}
Du er sikkert stødt på 2 lineære ligninger i to ubekendte. Et eksempel er
\[
  \arraycolsep=1.4pt%\def\arraystretch{2.2}
  \begin{array}{rcrcl}
    2x&-&y&=&-3\\
    4x&-&2y&=&-9
  \end{array}
\]
Vi ved hvordan man løser disse ligninger, hvis altså de har en løsning. Vi kan først isolere \(x\) i den første ligning:
\begin{align*}
  2x-y=-3\\
  \Rightarrow 2x=-3-y\\
  \Rightarrow x=\frac{-3}{2}-\frac{y}{2}.
\end{align*}
Vi kan sætte dette udtryk for \(x\) ind i ligning 2 og isolere \(y\).
\begin{align*}
  4\ps*{\frac{-3}{2}-\frac{y}{2}}-2y=-9\\
  \Rightarrow -6-2y-2y=-9\\
  \Rightarrow -4y=-3\\
  \Rightarrow y = \frac{3}{4}.
\end{align*}
Vi har fundet \(y\). Vi kan nu indsætte \(y\) i den første ligning igen for at finde \(x\).
\begin{align*}
  2x-\frac{3}{4}=-3\\
  \Rightarrow 2x=-\frac{9}{4}\\
  \Rightarrow x=-\frac{9}{8}.
\end{align*}
Denne proces er dog lidt nørklet. Desuden kan man overveje følgende:
\begin{quote}
  \emph{Hvad nu hvis man havde 3 lineære ligninger i 3 ubekendte? Eller \(n\) ligninger i \(n\) ubekendte for \(n\ge 3\)?}
\end{quote}
Godt nok kan man lave en lignende proces, men det bliver kun mere nørklet jo flere ligninger man tilføjer. Det er i denne slags situation at man som matematiker bør overveje
\begin{quote}
  \emph{Findes der en nemmere og mere generel måde at løse denne slags ligninger på?}
\end{quote}
Og, da matematikere er dovne, ønsker vi særligt en metode som kræver mindre tankevirksomhed. Heldigvis er svaret på vores spørgsmål et klart ``ja!'', og vi vil nu beskæftige os med en måde at gøre det på, kaldet \emph{Gau\ss-(Jordan)-elimination}.
\section{Matricer og Gau\ss-Jordan-elimination}
Vi betragter igen vores ligningssystem
\[
  \arraycolsep=1.4pt%\def\arraystretch{2.2}
  \begin{array}{rcrcl}
    2x&-&y&=&-3\\
    4x&-&2y&=&-9
  \end{array}
\]
I stedet for at skrive det op som ovenfor, kan vi skrive det op som en liste af tal på \emph{matrixform:}
\[
  \begin{pNiceMatrix}[first-col,first-row]
    & x & y & \\
    \text{ligning 1} & 2 & -1 & -3\\
    \text{ligning 2} & 4 & -2 & -9
  \end{pNiceMatrix}
\]
Hver \emph{række} i matricen repræsenterer en af ligningerne. Den første \emph{søjle} af matricen repræsenterer koefficienterne foran \(x\) i ligninerne, mens den anden repræsenterer koefficienterne foran \(y\). Den sidste søje repræsenterer konstantleddet i de to ligninger. Vi kunne også have skrevet matricen som
\[
  \begin{pmatrix}
    -1 & 2 & -3\\
    -2 & 4 & -9
  \end{pmatrix},
\]
hvor vi altså har byttet om på \(x\) og \(y\)'s plads: Ligningssystemet er stadig det samme; vi har bare byttet om på pladserne for variablerne. Så længe én bestemt kolonne altid svarer til \emph{præcis ét} variabel, så er det en gyldig måde at skrive systemet op på. Dog er den sidste række \emph{altid} den, som svarer til konstantleddene. Derfor skriver vi også nogle gange matricen som
\[
  \begin{pmatrix}[cc|c]
    -1 & 2 & -3\\
    -2 & 4 & -9
  \end{pmatrix}
\]
Helt generelt definerer vi nu matricer
\begin{definition}
  En \emph{reel} \(m\times n\) matrix, eller en matrix med \(m\) rækker og \(n\) søjler, er en liste af tal
  \[
    \begin{pmatrix}
      a_{11} & a_{12} &\ldots & a_{1n}\\
      a_{21} & a_{22} & \ldots & a_{2n}\\
      \vdots & \vdots & \ddots & \vdots\\
      a_{m1} & a_{m2} & \ldots & a_{mn}
    \end{pmatrix}
  \]
  hvor hvert \(a_{ij}\) ligger i \(\R\). Mængden af alle \(m\times n\)-matricer skriver vi som \(M_{m\times n}(\R)\). Hvis \(n=m\) skriver vi dog \(M_n(\R):=M_{n\times n}(\R)\).
\end{definition}
Vi har nu en generel måde at skrive \(n\) lineære ligninger i \(n\) variable op i en \(n\times (n+1)\)-matrix: Hvis vi har et system af ligninger
\[
  \arraycolsep=1.4pt%\def\arraystretch{2.2}
  \begin{array}{llllllllcl}
  a_{1,1}x_1&+&a_{1,2}x_2&+&\dots &+&a_{1,n}x_n&=&b_1\\
  a_{2,1}x_1&+&a_{2,2}x_2&+&\dots &+&a_{2,n}x_n&=&b_2\\
    &&&&&&&\hspace{2.5pt}\vdots&\\
  a_{n,1}x_1&+&a_{n,2}x_2&+&\dots &+&a_{n,n}x_n&=&b_m
  \end{array}
  \]
  får vi på naturlig vis en matrix
  \[
    \begin{pmatrix}[cccc|c]
      a_{1,1} & a_{1,2} & \ldots & a_{1,n} & b_1\\
      a_{2,1} & a_{2,2} & \ldots & a_{2,n} & b_2\\
      \vdots & \vdots & \ddots &\vdots & \vdots\\
      a_{n,1} & a_{n,2} & \ldots & a_{n,n} & b_n
    \end{pmatrix}.
  \]
  \begin{quote}
    \emph{Men hvad kan vi bruge det til?}
  \end{quote}
  Vi laver nu nogle generelle observationer om lineære ligningssystemer. Til det bruger vi igen vores modeleksempel
  \[
  \arraycolsep=1.4pt%\def\arraystretch{2.2}
  \begin{array}{rcrcl}
    2x&-&y&=&-3\\
    4x&-&2y&=&-9
  \end{array}
  \]
  Vi kan lægge \(-3\) til på begge sider af ligning nr. 2, og få en ækvivalent ligning.
  \begin{align*}
    4x-2y-3=-9-3\\
    \Leftrightarrow 4x-2y+(-3)=-12.
  \end{align*}
  Men vi kan bruge ligning 1 til at skrive \(-3\) om
  \begin{align*}
    4x-2y+2x-y=-12\\
    \Leftrightarrow 6x-3y=-12.
  \end{align*}
  Vores oprindelige ligningssystem bliver altså nu til det ækvivalente ligningssystem
  \[
  \arraycolsep=1.4pt%\def\arraystretch{2.2}
  \begin{array}{rcrcl}
    2x&-&y&=&-3\\
    6x&-&3y&=&-12
  \end{array}
  \]
  Det giver os en generel regel:
  \begin{quote}
    \emph{i et lineært ligningssystem kan man lægge én ligning til en anden ligning og få et ækvivalent ligningssystem.}
  \end{quote}
  Vi kan omsætte vores observation til matricen som svarer til ligningssystemet.

  Reskalering

  Bytte om
\section{At regne med matricer}
\end{document}
